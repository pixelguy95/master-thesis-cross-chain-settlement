\Subsection{Implementation specifics}
So let's take a closer look at how I implemented this. To work with bitcoin and litecoin you
need a very specific setup. I had one Bitcoin node and one Litecoin node running on my raspberry pi 3
at home. Both running the latest version of their respective core implementations. The nodes were used 
for interacting with the blockchain (broadcasting, reading, monitoring etc...). 
BOth of them ran on the testnet version of the respective blockchains.
This is a global test network where the coins are considered worthless,
but the important parts like blocktime and consensus works exactly the same.
THe testnet exists exactly for the purpose of testing contracts and other bitcoin software
before use in the real world.

My implementation was entirely writen in go-lang. I had never written any code in go
so at the start I spent a day or two just learning how to operate. THe reason 
for choosing go was mostly because of the many great libraries that are available
when programming towards bitcoin and other cryptocurrencies. The most used
library was the btcd-suite. This is the code used for the go-lang specific implementation
of a bitcoin node, it is modularized in such a way that you can basically use it for any
Bitcoin specific task. In my case most of the coding was in regards to constructing
the specialized transactions. 

The on-chain atomic swap implementation actually consists of three small programs.
Two of the programs are from ALices perspective and the other two are from
Bobs perspective. In my little example the bargening and agreement parts are
set in stone, meaning that the exchange rate, timelocks etc are hardcoded.
Both participants are represented as data structures that contains stuff like
their private key and their public key. 

The first program constructs the very first contract transaction that Alice
will broadcast to the bitcoin blockchain. THe btcd libarary provides the
proper datastructures for the transaction related stuff, the contract has 
to be constucted manually however. This is the entire contract script 
for Alice: *Show on screen*

None of the programs actually broadcast the transactions, instead
they print it in hex-format. THe hex-formatted transaction can be broadcast
manually by the user by running a simple command in the Bitcoin-core node.

The other programs are just part of the steps I described earlier. The second
program creates the Bob side of the contract, the contract looks exactly the same but
the timelock as been altered to 24 hours, and of course the payout is in 10 litecoins.
The third and fourth program constructs the claim transactions. 

Overall making this into several programs was pretty pointless. On the bright
side it works just as intended. I tested all possible scenarios and they
all worked just like intended. 
