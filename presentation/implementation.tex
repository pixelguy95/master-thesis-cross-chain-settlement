\chapter{Implementation of atomic swaps}
So finally we get to the implementation part of the presentation. Becauce I did
two different types of atomic swaps I have decided to split this part up into two parts.
The first parts cover what is known as on-chain atomic swaps, and the second part
will be the bit more complex off-chain atomic swaps.

\Section{on-chain atomic swap}
THere are several ways to implement a on-chain atomic swap. The one I choose 
is probably the easiest to understand. Before I go into implementation specifics 
it would be good to know how this process actually works. 

Imagine two people, Alice who has Bitcoins and wants litecoins, and Bob who
has litecoins and wants bitcoins. Here a swap is possible. THe methods people
start thinking about right away could be for example some sort of third-party
that ahndles the swap. Or maybe Alice just sends the Bitcin to Bob and hopes
that he doesnt take the money and run. Overall the problem he is trust. 
Even with a third-party there is a chance that Bob and the third-party
are cooperating to steal from Alice. With the help of programmable contracts 
however, we now have a method of swapping where the only thing you have to 
trust is numbers.

The process of an on-chain atomic swap is as follows:
Alice and Bob agrees to do a swap, 1 bitcoin for 10 litecoins. They also 
decide that Alice should be the one to initiate the exchange. 
To start off, Alice generates a random bytestring that will act as a 
pre-image, let's call this $R$. She hashes this pre-image and produces $H_R$.  
With te help of $H_R$ she constructs a new swap contract with the following clauses. Pay 1
bitcoin to Bob if he can provide the pre-image of $H_R$ ($R$). If Bob does not claim 
this output wihin 48 hours, refund the full amount to Alice. 

Alice broadcasts this contract transaction to the bitcoin blockchain and
notifies Bob of doing so, she also sends the unhahsed contract to Bob. 
Bob can then fetch the transaction from the blockchain, he then makes
sure that the contract hash matches the one on the chain (P2SH) and
he validates all the details of the contract. 

If something is wrong here or if Bob changed his mind, he does not have 
to do anything, and Alice can refund her money after 48 hours. 
He can not try to claim Alices bitcoins as he does not know the
pre-image.
If everything in the contract is alright however he constructs a new contract with 
the following details. Alice gets 10 litecoins if she can provide
the pre-image to $H_R$, if she does not claim them within \textbf{24 hours}:
refund the full amount to Bob. Bob broadcasts this transaction to the litecoin
blockchain then notifies Alice and sends over the unhashed contract. 

Alice fetches the contract and validates all the information just like
Bob did. If something is wrong or she has changed her mind she can do nothing,
Bob will get his refund after 24 hours and Alice gets her refund after 48 hours.
Otherwise she can claim Bob's litecoins, she does this by constrcting a claim
transaction containing the pre-image and a signature from her private key. 

Meanwhile Bob has been monitoring the litecoin blockchain and when Alice claims the
litecoins he will see the transaction together the pre-image data. Now when he has
the pre-image he can construct his own claim transaction that claims the bitcoins 
from the initial contract. 

The safety of this swap comes from the reveal mechanism of the pre-image being
the blockchain and the decreasing timelocks. Those of you who have paid close
atention will notice that the mechanisms are very much like the ones used in hashed
timelocked contracts. 

\Subsection{Implementation specifics}
So let's take a closer look at how I implemented this. To work with bitcoin and litecoin you
need a very specific setup. I had one Bitcoin node and one Litecoin node running on my raspberry pi 3
at home. Both running the latest version of their respective core implementations. The nodes were used 
for interacting with the blockchain (broadcasting, reading, monitoring etc...).