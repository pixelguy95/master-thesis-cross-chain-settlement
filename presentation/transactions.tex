\Subsection{Script \& Transactions}
Alright, now that we got consensus and the cryptography down let's take a closer look at how programmable contracts
are made possible over the bitcoin network.

Built into Bitcoin there is a programming language called Script. Not the best of names,
but let's not think about that for now. Script is the code that defines exactly what a transaction can and
cannot do. It is a very bare bones language, with byte sized instructions, no heap memory,
and only a very limited stack memory, that strictly operates like a stack.
Very reminiscent of the Forth language.

A piece of Script code can be defined as either valid or invalid. The only
way a script could be valid is if at the end of executing the last instruction
there is only one value on the stack, and it equates to true. Anything else
will make the script invalid, in fact under certain circumstances the
script could be marked as invalid before the execution is even finished.

Script has one major limitation compared to other programming languages, it is by design
not Turing complete. Basically meaning
that it cannot do anything. What is missing from the usual programming language repertoire
is loops. Script has no construct that allows for repeated instruction execution.
Instead all Script programs are completely linear and always execute in a limited time.
The reason for this has to do with the halting problem and a bunch of anti-spam
measures. Even with this limitation Script has a very wide use case as you will soon see.

Transactions in Bitcoin are not as straight forward as you might expect.
It's not just a matter of moving from one attached name to another.
A transaction in Bitcoin is a datatype consisting of meta data, a list
of inputs and a list of outputs. An input is a reference to a previous
output from an earlier transaction, an output could be seen as a destination,
it holds the amount sent and the ''destination''. Because of
the references to previous transactions in the input you can
follow transaction all the way back to the generation
transaction when the coins were created in the first place.

The inputs and outputs are where Script comes into the picture.
A output contains a partial script. A input contains the
complementary part forming the entire script. This is what makes Bitcoin
transactions so versatile. You could make the output script
be anything that is describable in Script.

For example *Show simple script on screen*. This script requires the redeemer
to give the answer to this equation to spend the money in the output.
*Show redeeming input script*.

Obviously most Bitcoin transactions are not this basic. The most common
script type has a name, P2PKH, Pay to public key hash. In this script
the output is given a hashed public key, and the one who wants to
spend the output has to provide a signature from the private key related to the
given public key. *Show script and redeeming script*

A slightly newer development in Bitcoin is P2SH, Pay to script hash.
If the previous example can be seen as paying to a bitcoin address,
this one can be seen as paying to a script, or a contract.
In this case anyone who can provide the contract together with
a partial script that makes the contact valid, is allowed to redeem.
The output paying to the script contains only the hash of the script,
as the name implies.
