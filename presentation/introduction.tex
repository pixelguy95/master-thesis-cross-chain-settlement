\chapter{Introduction \& Background}
Alright, so the thing that is on everybody's mind is probably: what is an atomic 
swap? An atomic swap is where two parties exchange assets atomically, which means
that either the transaction takes place fully, or the state is reset to the pre-
exchange state. 

*Read from oracle defenition*

This is made possible by clever use of cryptography and 
programmable contracts on the bitcoin network and blockchain. So before we can 
go into more detail on this I should first cover the basics of Bitcoin.

So, most people, especially in computer science, have heard of Bitcoin, but I could 
almost count on one hand the number of people I have met that have more than 
a basic understanding of how it works. I could talk for hours about this 
subject, but sadly there is no time for that. So I will try to give you the 
shortest possible version where you can at least understand the rest of my
thesis.

The simplest description of Bitcoin is a shared public ledger, that relies 
on proof-of-work to build network-wide consensus. First of proof-of-work
is a way to prove mathematically that work was put into doing something. 
The most common way of doing this is via hashing of some datatype.
The hash has to meet certain criteria to be accepted. There is no known 
way of producing a wanted hash, so the only way is to try different combinations
until a good result is found. So if you have data that produces a certain hash
that hash serves as proof that you put work into creating it.

Another thing you have probably heard about before is the blockchain, but
just as with Bitcoin overall, people know little about what it actually is. 
A blockchain is basically a shared datatype. It is very reminiscent of a
linked list, but allows for branching, meaning that two elements can link 
to the same parent. We will come back to this in a moment, but first, let's 
take a closer look at the blocks.

A block is a data structure that has a header and data. The header contains 
metadata about the block itself as well as a reference to the previous block
in the chain. In Bitcoins case, the data in the block is just a list of transactions,
but you could put anything you want into this field. The reference to the previous
block is what forms the chain. You can from any block follow the references 
all the way back to the original block. Anyone can add a block to the chain. But 
it has to meet the proof-of-work criteria. The Bitcoin network independently calculates 
something called mining-difficulty. This is represented by a large 256-bit number. 
For your new block to be accepted the header of the block has to produce a hash that
is strictly smaller than the difficulty number. You produce unique hashes by changing 
a field in the header called nonce, this process is what is referred to as mining. 

The mining-difficulty is set so that the sum of all participant's hash-calculating power,
or hashrate, will produce  new block on average every 10 minutes. The difficulty is adjusted 
every 2016th block. 

So how does proof-of-work ensure that the shared ledger stays consistent? This is
where concepts like longest chain comes in. The Bitcoin network only accepts the longest
chain as truth, in other words the chain with most accumulated proof-of-work. 
This works as long as the majority of participants is honest. However due to the
probabilistic manner of how new valid blocks are found there is still a chance for 
contention even if all participants are honest. For example what will the network do
in the case where two different blocks are found at almost the same time, in 
different parts of the network? Now 
there are two chains of equal length. So how does the network decide which one
is correct?

*Explain pictures*

%https://en.bitcoin.it/wiki/Transaction
%mastering bitcoin
\Section{Transactions}\label{transactions}
Transactions in Bitcoin are not as straight forward as you might expect a transaction to be. A transaction contains a list of inputs and a list of outputs as well as some metadata like version number and lock-time. 

\begin{figure}[H]
	\centering
	\includegraphics[width=1.0\textwidth]{background/images/transaction_basic.png}
	\caption{4 example transactions and how inputs are connected to outputs}
	\label{fig:transaction_input_output}
\end{figure}

In simplified terms an output could be seen as the destination of a transaction, in other words it says how much and to whom the transaction is sent to. An input is a reference to a previous output. The inputs take the money from the outputs they reference and that money is used to fun the new outputs. 

The inputs and outputs is where Script comes into the picture. Both outputs and inputs contains an incomplete script, together however they complete the script. The script in an output could be seen as a challenge, and the script in the input is the response. When a transaction is tested for validity the input script is appended to the script in the output and is executed. If the script comes out as valid the transaction is also valid. Here is a basic example: Let's say Alice wants to send a transaction to whoever can answer the very complex equation $4+3$. Her transaction output would contain the script:

\texttt{4 3 OP\_ADD OP\_EQUALS}

If this is executed as is it is invalid. But let's say Bob knows the answer to the equation he can then create a new transaction where the input contains the script: 

\texttt{7} 

Just as before this script is not valid by itself. But then the transaction is checked for validity the input will be appended to the start of the output script forming the following: 

\texttt{7 4 3 OP\_ADD OP\_EQUALS}

Which is a valid script, thus bobs new transaction is also valid and he may spend the money as he see fits. 

Obviously most transactions on the Bitcoin blockchain are not this simple. The most common form of transaction contians a script called P2PKH which stands for Pay to public key hash. Before we can go into details on this one however we first need to know about how signatures and sighash work in script and transactions.

\Subsection{Signatures and sighash}
Section \ref{ecdsa} covers public keys and signatures in depth.

Perhaps the most important operation in script is the \texttt{OP\_CHECKSIG} operation and its cousins. \texttt{OP\_CHECKSIG} pops two values from the stack, if the script is correctly implemented these two values should be the public key and a signature that was signed with the private key that was used to create the public key. 

The question is: what is signed when the signature is created? Broadly speaking it is the hash of the transaction that is trying to spend the output, this is not entirely correct however. Appended to the signature that is a flag called \textbf{sighash}. The value of sighash tells the script interpreter what hash was signed during the creation of the signature. There are 4 types of sighash implemented:

\Subsubsection{SIGHASH\_ALL}
This can be considered the default sighash, if it is not specified it can safely be assumed that this type was used. This simply means that the entire transaction is signed with all outputs and all other inputs.


\Subsubsection{SIGHASH\_NONE}
This one signs the transaction butwithout the outputs, it could be thought of as ''I don't care where the money goes''

\Subsubsection{SIGHASH\_SINGLE}
All outputs are removed except the output with the same index as the input that is being signed, then that transaction is signed.

\Subsubsection{SIGHASH\_ANYONECANPAY}
Signs the transaction with all the outputs but none of the other inputs. This basically means ''The money has to go here, but I don't care if someone else want to fund this transaction also''

\Subsubsection{More detailed process}
On the next page the entire signing process for SIGHASH\_ALL is detailed. This is how it is performed in the actual implementation:
\newpage
\centerline{\includegraphics[width=1.35\textwidth]{background/images/checksig_in_detail.png}}
\newpage



\Subsection{Pay to public key hash (P2PKH)}
\Subsection{Pay to script hash (P2SH)}
\Subsection{Timelock and sequence}
