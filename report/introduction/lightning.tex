
\Section{Payment channels}
One of the biggest problems facing bitcoin is scalability. At time of writing
onchain transactions is capped at about $\approx 7$ T/s (transactions er second).\cite{scaling}
This has to do with network propagation and the hard cap on block sizes, a block
can only contain so many transactions before it is full. There are a couple of
propsed solutions to this however, and one of them is connected payment channels

First off, a payment channel in bitcoin is a type off trick using programmable transactions, 
were two users can open a bidirectional payment channel where an infinite number of transactions 
could be trustlessly exchanged without using the blockchain. 

A channel can be opend by one or both participants by using a funding transaction. The funding 
transaction requires the signature of both participants to spend and is transmited to the blockchain. 
After that the participants in the channel exchange commitment transactions that represent exchange in money.
Any of the commitment transactions could be broadcast to the blockchain whenever a participant 
want to close the channel. Once a channel has been closed it cannot be used for further transactions.
Clever mecahnizms exists in the creation of these channels that makes it so one participant 
can't cheat the system and spent moeny that does not belong to them.

Just payment channels alone were not enough to fix the scaling problem however, as a payment channel 
only allows two parties to exhange unlimited transactions. A proposed extension to the 
payment channels is the lightning network.

\Subsection{Lightning network}
Lightning network is a relatively recent development in the bitcoin community.
Payment channels has been known about for a while. But in
January of 2016 a white paper was released detailing a promising new extension.\cite{lightningnetwork_2019}
It showed that with a few changes to the bitcoin protocol a new type of
payment channel could be opened that allows transactions to propagate through multiple channels.\cite{lightningnetwork_2019}

\begin{figure}[H]
	\centering
	\includegraphics[width=0.70\textwidth]{introduction/images/mesh_network.png}
	\caption{A basic overview of a lightning network, each node represent someone
	and each edge represents a channel between two people.}
	\label{fig:mesh}
\end{figure}

Lightning network is really just a network of peers connected via payment channels. In figure \ref{fig:mesh} is an example. Let's say that Alice (node A) want to send a traansaction to Qbert (node Q) but they have no direct payment channel between them. With lightning network they can send the transaction via the peers that are between them. 