\chapter{Implementation}
To implement the described types of atomic swaps, Go (golang) was used 
together with the standard libraries plus external libraries from the 
\textbf{btcsuite}: \textbf{btcsuite/btcd} and \textbf{btcsuite/btcutil}, 
as well as \textbf{lightningnetwork/lnd}, all of which can be found 
on github.com. The implementations are simplified, just like with 
this report the focus was on making it easy to understand for newcomers. 

To interact with the different blockchains two different full-nodes 
was setup. \textbf{Bitcoin Core 17.1} for bitcoin and 
\textbf{Litecoin Core  v0.16.3.0-82cacfef8}. Both running on their 
respective \textit{testnet}. Both nodes have a RPC interface to interact 
with the blockchain and the rest of the network. 

\Section{On-chain atomic swaps}
The implementation creates two simulated peers, called Alice and Bob. 
Each participant is represented by the wallet name used in the node
programs. The entire process is divided up into 4 smaller programs.
The first program generates the contract and contract transactions and 
prints it on \texttt{stdout} in hex format. The second program takes 
the contract as input and prints the contract transaction for the 
litecoin side of the swap. 

The last two programs are used to redeem the contract transactions.
Logically you should run the litecoin side program first and
claim the litecoins as Alice knows the secret. The third program
takes the secret as input and prints the transaction that can
claim the litecoins. The fourth program takes the secret and creates 
the claim transaction for the bitcoins. 

Overall the on-chain swap implementation is pretty sloppy. 
In the off-chain swap implementation everything was put into one program, with 
clearly defined datatypes and it relies a lot less on the nodes for support 
with private keys and public keys.

The swapping programs does not themselves send any transactions to the blockchain,
that has to be done by the user manually. 

\Subsection{Details on script}
The script used to create the contract is based on the suggested atomic-swap 
contract that can be found in \textbf{btcd}. More specifically there 
is a function that comes with the btcd library that can extract data from 
swap contracts, so the contract designed in this solution was designed so that
it is compatible with this function. 

\begin{verbatim}
OP_IF
    OP_SIZE
    <Byte length of secret>
    OP_EQUALVERIFY
    
    OP_SHA256
    <Hash of secret>
    OP_EQUALVERIFY
    
    OP_DUP
    OP_HASH160
    <Receiver address>
OP_ELSE
    <Locktime>
    OP_CHECKLOCKTIMEVERIFY
    OP_DROP
    
    OP_DUP
    OP_HAASH160
    <Refund address>
OP_ENDIF

OP_EQUALVERIFY
OP_CHECKSIG
\end{verbatim}

A contract following this format can be used by the contract details 
extractor function to extract all the data pushes, like locktime and 
hash of secret.

\Section{Off-chain atomic swap}
The implementation of the off-chain swap was considerably more robust and better coded.
The solution creates two fake people Alice and Bob, who via two payment channels performs
an atomic swap from bitcoin to litecoin (From Alice's perspective). Both the peers and
channel has proper data structures  and related functions. The process starts with 
opening of both channels. The channel opening consists of generating the funding transaction 
and the first commitment transaction. Just as before they are printed to \texttt{stdout} 
in hex encoding, and also like before all transactions has to be broadcast manually. In
this case however you dont really need to transmit any transactions until the channel closes,
The commitments coming after the first ones assume that the funding transaction has been broadcast,
but they never check for it. So you could run the simulation without actually performing any transactions
on-chain.

The payment channels in this implementation supports regular transactions between the peers. The implementation
is an exact clone of how it works in the lightning network minus all the protocol and multi-hoping.
Some liberties are taken with how data is stored and the order of how things are done. For example 
all data related to things like private keys are stored on the same computer. In a real life scenario this
would be very unsafe, but making the experiment using two computers would be needlessly complicated, so
it was skipped. An example of things being out of order is the revocation of commitment transactions, which 
in this solution is done at the same time as the commitment itself. This is also completely unsafe 
in any real life scenario, all money would be stolen if any commitment transaction were to be broadcast. 

The implementation has a special function where the real swapping magic takes place. 
It takes in two channels, the ones we created earlier, and performs the atomic swap between them.
The regular channel payment function that is contained in the payment channel object
takes in a payment descriptor that takes in information about the transaction, the 
descriptor holds data like amount, sender and receiver, but also the pre-image hash
that should be used plus the timelock.

The swap function performs the swap by first sending bitcoins from Alice to Bob, then litecoin 
from Bob to Alice using the same pre-image hash and with a halved time in the future in the timelock.
It then simulates pre-image reveal and settles both channels with the updated value.

All outcomes of the transaction is tested. That means all possible scenarios and outcomes from 
the htlc contract on both channels. The test was by trying all scenarios by broadcasting 
the commitment transactions on the testnet chains along with all transactions that settles the htlc
and other outputs. 

\Subsection{Implementation specifics}
Both the channels used a variation of the implementation that can be found in lnd. With a few modifications
to allow the custom channel setup. As mentioned none of the BOLT protocols and no databases were used (Data exists only while the program is running). 

All the scripts for the different contracts were as described in the BOLT specification. 
Because of an initial misunderstanding there is a second-level htlc on the htlc-success path
and the timeout-path on both sides of the commitment transaction. This is not correct, but it
does not affect the security or the swaps in any way so it was left in.