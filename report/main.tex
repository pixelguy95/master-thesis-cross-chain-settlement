%\documentclass[a4paper,11pt]{IEEEtran}
\documentclass[11pt, titlepage, oneside, a4paper]{report}
\usepackage[english]{babel}
\usepackage[utf8]{inputenc}
\usepackage[x11names]{xcolor}
\usepackage{listings} % http://ctan.org/pkg/listings
\usepackage{url}
\usepackage{lastpage}
\usepackage{datetime}
\usepackage{fancyhdr}
%\usepackage{standalone}
%Includes "References" in the table of contents
\usepackage[nottoc]{tocbibind}
\usepackage{varwidth}
\usepackage{verbatim}
\usepackage{graphicx}
\usepackage{float}
% type-set color
\usepackage{amsmath}
\usepackage{amssymb}
\usepackage{hyperref}
\usepackage{todonotes}
% psudeocode
\usepackage{clrscode3e}
\usepackage{algorithm}
\usepackage{newfloat}
\usepackage{listings}
\usepackage{color}

\usepackage{enumitem}
\usepackage{subcaption}
\usepackage{multicol}
\usepackage[top=50pt,bottom=50pt,left=85pt,right=85pt]{geometry}
\usepackage{float}
\usepackage{wrapfig}
\usepackage{framed}
\usepackage{lipsum} % for dummy text only
\usepackage[alpine,misc]{ifsym}

\input{insbox.tex}



\definecolor{mygreen}{rgb}{0,0.6,0}
\definecolor{mygray}{rgb}{0.5,0.5,0.5}
\definecolor{mymauve}{rgb}{0.58,0,0.82}
\newenvironment{centerverbatim}{%
  \par
  \centering
  \varwidth{\linewidth}%
  \verbatim
}{%
  \endverbatim
  \endvarwidth
  \par
}

\DeclareFloatingEnvironment[placement={ht!},name=Grammar]{grammar}
\DeclareFloatingEnvironment[placement={ht!},name=List]{mylist}
\DeclareFloatingEnvironment[placement={ht!},name=Language]{lang}
\setlist[itemize]{noitemsep, topsep=0pt}

\def\bitcoinA{%
	\leavevmode
	\vtop{\offinterlineskip %\bfseries
		\setbox0=\hbox{B}%
		\setbox2=\hbox to\wd0{\hfil\hskip-.03em
			\vrule height .3ex width .15ex\hskip .08em
			\vrule height .3ex width .15ex\hfil}
		\vbox{\copy2\box0}\box2}}

\newcommand\invisiblesection[1]{%
	\refstepcounter{section}%
	\addcontentsline{toc}{section}{\protect\numberline{\thesection}#1}%
	\sectionmark{#1}}

\lstset{ %
  backgroundcolor=\color{white},   % choose the background color; you must add \usepackage{color} or \usepackage{xcolor}; should come as last argument
  basicstyle=\footnotesize,        % the size of the fonts that are used for the code
  breakatwhitespace=false,         % sets if automatic breaks should only happen at whitespace
  breaklines=true,                 % sets automatic line breaking
  captionpos=b,                    % sets the caption-position to bottom
  commentstyle=\color{mygreen},    % comment style
  deletekeywords={...},            % if you want to delete keywords from the given language
  escapeinside={\%*}{*)},          % if you want to add LaTeX within your code
  extendedchars=true,              % lets you use non-ASCII characters; for 8-bits encodings only, does not work with UTF-8
  frame=single,	                   % adds a frame around the code
  keepspaces=true,                 % keeps spaces in text, useful for keeping indentation of code (possibly needs columns=flexible)
  keywordstyle=\color{blue},       % keyword style
                   % the language of the code
  morekeywords={*,...},            % if you want to add more keywords to the set
  numbers=left,                    % where to put the line-numbers; possible values are (none, left, right)
  numbersep=5pt,                   % how far the line-numbers are from the code
  numberstyle=\tiny\color{mygray}, % the style that is used for the line-numbers
  rulecolor=\color{black},         % if not set, the frame-color may be changed on line-breaks within not-black text (e.g. comments (green here))
  showspaces=false,                % show spaces everywhere adding particular underscores; it overrides 'showstringspaces'
  showstringspaces=false,          % underline spaces within strings only
  showtabs=false,                  % show tabs within strings adding particular underscores
  stepnumber=1,                    % the step between two line-numbers. If it's 1, each line will be numbered
  stringstyle=\color{mymauve},     % string literal style
  tabsize=1,	                   % sets default tabsize to 2 spaces
  title=\lstname                   % show the filename of files included with \lstinputlisting; also try caption instead of title
}


\hypersetup{colorlinks,linkcolor={blue},citecolor={blue},urlcolor={blue}}
\makeatletter
\newcommand\footnoteref[1]{\protected@xdef\@thefnmark{\ref{#1}}\@footnotemark}
\makeatletter

%Title, date an author of the document
%Define special date YYYY-mm-dd
\newdateformat{specialdate}{\THEYEAR-\twodigit{\THEMONTH}-\twodigit{\THEDAY}}
\date{\specialdate\today}
\def\datemade{\specialdate\today}

%Define paper
\def\university{Umeå University}
\def\instution{Department of Computing Science}
\def\pagetypename{Master thesis report}

%Define course
%Examensarbete för civilingenjörsexamen i teknisk datavetenskap
\def\coursename{Master degree project in computer science}
\def\coursecode{5DV143}
\def\coursegiventerm{VT19}
\def\coursepoints{30HP}
\def\titleFrontPage{\coursename\\\coursepoints, \coursecode, \coursegiventerm}
\def\usupervisor{Jan-Erik Moström (\url{jem@cs.umu.se})\\}
\def\csupervisor{Oskar Jansssssson (\url{oskar.janson@cinnober.com})\\}

%Define assignment
\def\assignmentname{Evaluating Cross-chain
	Settlement and Exchange in
	Cryptocurrency}

%Labpartners (separate with comma)
\def\csuser{c14can}
\def\casuser{caan0156}
\def\name{Carl-Johan Andersson}

%Header
%\setlength{\headheight}{15pt}
\setlength{\headheight}{27.11652pt}
\setlength{\headsep}{20pt}

\pagestyle{fancy}
\fancyhead[LE,RO]{\datemade\\\name}
\fancyhead[RE,LO]{\coursename\\\university}
\fancyfoot[CE,CO]{\rightmark}
\fancyfoot[LE,RO]{\thepage}
\renewcommand{\headrulewidth}{1pt}
\renewcommand{\footrulewidth}{1pt}

%Sets the paragraph formatting
\setlength{\parindent}{0pt}
\setlength{\parskip}{10pt}

\begin{document}
%Begining of the documents

%Frontpage
\begin{titlepage}
	\thispagestyle{empty}
	\noindent {\large \MakeUppercase\university \\
				\instution \\
				\pagetypename \\
			  }

	\begin{center}
	\Large{\textbf{\titleFrontPage}}\\[7pt]

	\Large{\assignmentname}\\[40.0pt]
    
	\end{center}
	\thispagestyle{empty}
\end{titlepage}

\onecolumn
\newpage
% TOC - Table Of Contents
% \setcounter{secnumdepth}{2}
% \setcounter{tocdepth}{2}
\pagenumbering{Roman}

\newcommand{\Section}[1]{\section{#1}\vspace{-15pt}}
\newcommand{\Subsection}[1]{\vspace{-4pt}\subsection{#1}\vspace{-15pt}}
\newcommand{\Subsubsection}[1]{\vspace{-4pt}\subsubsection{#1}\vspace{-15pt}}

\chapter*{Glossary}\vspace{-10mm}
\textbf{Crypto currency} - A currency backed not by centralized authorities but by mathematical evidence and clever mechanisms\\
\textbf{Bitcoin} - The very first and most mature cryptocurrency on earth, 1 bitcoin = $1$\bitcoinA\\
\textbf{Litecoin} - A alternative to Bitcoin, mostly a clone with a few changes\\
\textbf{satoshi} - The smallest fraction of a bitcoin. $100.000.000$ satoshis = $1$\bitcoinA\\ 
\textbf{Satoshi Nakamoto} - The pseudonym used by the creator of bitcoin\\
\textbf{Proof-of-work} - A system where someone can prove mathematically that work was put into doing something.\\
\textbf{Mining} - In the context of crypto-currency refers to looking for a valid hash in a proof-of-work system. Most often by checking random numbers in the nonce field until a valid hash is found\\
\textbf{Onchain} - Something that is onchain (for example a transaction) has been included in the blockchain and can be safely assumed to be immutable, in other words can't be changed.\\
\textbf{Offchain} - Something that is not on the blockchain.\\
\textbf{Atomic (Adjective)} - Something that only has two outcomes. Either completed fully or no changes to the state at all. An atomic task can not be half completed.\\
\textbf{Alice and Bob} - Alice and Bob are two fictional people used in examples.\\
\textbf{Hash} - A hash is a mathematical way of producing a one-way ''fingerprint'' of any data, easy to check but nearly impossible to reverse.\\
\textbf{Pre-image} - A pre-image is the name given to data as it was before being altered. For example Hash(D) = H, in this case D would be the pre-image of H\\
\textbf{Nonce} - A temporary value, most often a number. Used to give the hash of a block a unique outcome. The value fills no othe purpose other than altering the hash. \\

\vspace{-5mm}
\section*{Payment channel specific glossary}\vspace{-5mm}
\textbf{Node} - A participant in a payment channel or lightning network. Sometimes goes under names such as peer, agent, actor or participant.\\
\textbf{Channel} - A construct where two participants can exchange transactions outside of the blockchain. The medium of exchange caan be assumed to be over an internet connection, but could technically be over almost any medium. \\
\textbf{Lightning network} - A lightning network is a network of connected payment channels. Where payments can be routed across multiple nodes and channels safely\\
\textbf{HTLC} - Hashed time-locked contract, a contract defining the outcomes of a payment over multiple channels.\\

\tableofcontents
\thispagestyle{empty}
\newpage
\pagenumbering{arabic}
\setcounter{page}{1}
\setcounter{section}{0}



%See introduction.tex

\Section{Introduction}
Although blockchain and other types of distributed ledgers are still in their infancy a growing number of people and companies are starting to see that they hold great promise, especially when it comes to financial technology, where things like trust and security is held to be very important. 

%https://nordic.businessinsider.com/bitcoin-history-cryptocurrency-satoshi-nakamoto-2017-12?r=US&IR=T
%https://bitcoin.org/bitcoin.pdf
%https://www.blockchain.com/sv/btc/block-height/0
%https://www.theverge.com/2013/5/6/4295028/report-satoshi-nakamoto
%bloody: https://bitcointalk.org/index.php?topic=234.msg1976#msg1976
%https://pastebin.com/syrmi3ET

%There is alot more things that could be writen here. speculations on who satoshi is, previous bitcoin like projects, why he diapeared etc...
\Subsection{The origins of bitcoin}
The most well known, developed and researched blockchain technology is know as Bitcoin. The mysterious nature of bitcoins creator makes hard to pinpoint how and when the idea was first thought up and when the development started, the most exact way and most well known would be to pinpoint at: \texttt{2009-01-03 18:15:05}, which is the timestamp on the very first block in the bitcoin blockchian, however the white paper (\textbf{Bitcoin: A Peer-to-Peer Electronic Cash System}) specifying the technical details circulated on cryptographic mailing lists as early as 31 October 2008, and the domain name \texttt{bitcoin.org} was registered 18 August 2008. 

\Subsubsection{Satoshi Nakamoto}

The author name given in the white paper is \textbf{\texttt{Satoshi Nakamoto}}, this name is believed to be a pseudonym. Satoshi remained in the bitcoin community for a couple of years. Regularly posting on the forum \texttt{bitcointalk.org} and keeping up with conversations in the mailing list. Those who have been interested in finding out Satoshi's real identity have analyzed his active time and language used. The findings were that Satoshi was most active during Western European day time, and he also used a lot of Anglo-colloquialisms such as ''bloody hard'', and ''flat'' instead of apartment, so a popular theory is that Satoshi lived in Britain at least during this time. 

\texttt{April 23, 2011} was the last time anyone ever heard from Satoshi Nakamoto, in a mail to a fellow developer Mike Hearn he said ''I've moved on to other things.  It's in good hands with Gavin and everyone.''. Speculations on who Satoshi really is still going strong even to this day, but Nakamoto's true identity is so far unknown. 


%See background.tex

\chapter{Bitcoin and Smart contracts}

This section is far from finished. 

\Section{Bitcoin: a peer to peer electronic cash}
As described in the title of the original white paper bitcoin is a peer to peer electronic cash system, which does not rely on any centralized third-party to neither verify the validity of any transactions or handling of transaction completion. Instead it is entirely decentralized and trust-less. The mechanisms and mathematics that makes this possible are relatively recent discoveries in computer-science. 


%This could be expanded a bit
\Subsection{Network wide consensus}
One of the main problems facing decentralized currencies before bitcoin was thought up is the byzantine generals problem (Byzantine fault). This refers to independent agents in a system being unable to reach a consensus on what has transpired and what actions to take next. The problem can be imagined as the network being split, where one subsection of the network believes transaction order \textbf{A} is correct while another non-overlapping subsection thinks transaction order \textbf{B} is correct. How can this be resolved, and how will a new node joining the network know which order is the right one.

Bitcoin was the first cryptocurrency to properly solve this problem once and for all, with the help of something called proof-of-work. proof of work originates from the slightly older idea of hashcash. 

Hashcash was/is a means to limit email spam and DDoS-attacks by a proof of work system. For example a sender could be required to produce a hash of message and nonce with a certain number of bits set to 0 at the start of the hash sequence. This (statistically) should take several attempts to produce. But correctness could be checked in a single step. Thus the hash sent together with the message could be considered proof of work, because there is no known way to produce such a valid hash of a message without trying it randomly, thus the only realistic way to have such a hash is if you worked for it. Looking for a valid hash in this kind of proof of work systems is often referred to as \textbf{mining}.

Bitcoin used this proof of work concept for another purpose however. Rather than combating spam the proof of work mechanism is used for reaching consensus. When ever a new block is added to the chain it needs to meet a certain proof of work requirement, called difficulty. This difficulty is set so that it should take the combined hashing power of all participants on average 10 minutes to find a valid hash of the next block in the chain.

There is a term called longest proof of work in bitcoin, this refers to the chain of blocks that has the most hash-power supporting it. The chain with the most work done is statistically the one with the most participants. As long as 51\% or more of the participants in the network are honest the longest chain can be trusted. So any new members can accept the chain with most work as the truth. 

%https://bitcoin.org/en/alert/2013-03-11-chain-fork
%https://gist.github.com/anonymous/3635514
%https://bitcoin.stackexchange.com/questions/3343/what-is-the-longest-blockchain-fork-that-has-been-orphaned-to-date
%https://github.com/bitcoin/bips/blob/master/bip-0050.mediawiki

\Subsubsection{Splits}
Contention for the longest chain can arise if a new block is found in two different parts of the network at (almost) the same time. This is not a problem and will eventually be resolved. If you imagine two blocks (\textbf{A1} and \textbf{B1}) being mined in different parts of the network with the same parent block and half the network got \textbf{A1} first and the other half of the network got \textbf{B1} first. While the entire network will accept all valid blocks they will only mine towards continuing the chain on the block they received first. So if a new block \textbf{A2} with the parent block \textbf{A1} is found first the chain formed by block \textbf{B1} will be considered invalid and the network continues the chain on the \textbf{A} side.\\

\InsertBoxR{0}{
	\footnotesize\setlength\fboxsep{10pt}\setlength\fboxrule{1pt}
	\fcolorbox{IndianRed3}{SlateGray1}{\begin{minipage}{2.1in}
			\invisiblesection{\textit{Side Bar}}
			\subsection*{How long was the longest split?}
			The longest splits that occurred by chance were 4 block long and has occurred at least at 3 different occasions. \\\\The longest split ever was caused by an update to the bitcoin core reference implementation (\textbf{0.8.0}) that rejected a block that the other implementations did not reject, the nodes accepting the new block keept building on it while those who had updated built on a different chain. The split lasted for 52 blocks before it was resolved.
	\end{minipage}}
}[7]


Such contention is called a split, or a fork, and happens naturally once every week or so. As explained they will eventually resolve themselves. The split becomes increasingly more unlikely to survive the longer it goes on. To begin with it is unlikely that a new block will form a split in the first place, and for the split to survive another block both new chains will have to receive a new block at almost the same time.

In common bitcoin lingo there is no difference between a split and a fork. But in this report a \textbf{split will referred to as one occurring unintentionally} and \textbf{a fork being intentional} just so there is no confusion

\Subsubsection{Forks}
Forks happen when there is a disagreement in the bitcoin community when it comes to protocol and consensus. The most famous fork in all of cryptocurrency occurred on \texttt{1st August 2017} and was caused by a conflict regarding the size of blocks. How bitcoin will scale to world-wide use has always been a hot debate in the bitcoin community, the majority seeks to scale bitcoin via second layer solutions such as payment channels, lightning network and side-chains. However there were those who disagreed, and instead wanted the network to handle more transactions by making the blocks larger at cost of centralization. 

A change in the blocksize requires the entire network to upgrade to the new protocol rules, but the block size increase was not accepted by enough nodes. So the group decided that a fork was the only way to resolve the issue. So as mentioned on \texttt{1st August 2017} the first block was mined on the new chain. 

The new chain got the name \textbf{Bitcoin cash} (bcash) while the main chain still is called just \textbf{Bitcoin}. Most miners and node owners stayed with Bitcoin, but a fraction jumped ship and started working on bcash instead. Today both chains runs along side each other, co-existing.

Forks are considered a valid way to vote on what consensus rules and protocol changes should be made. When a change is planned to be made in the chain those who disagree can branch off and follow the old rules. Thus in a way everyone gets their way. You could even branch off on your own forked-chain alone. The main problem for those forking is that most vendors and users still consider the largest (most users, most developers, largest price, etc...) chain to be the valid one. 

\Subsubsection{Soft and hard-forks}\label{soft_hard_fork}
TODO

%might be cut form the full report

%https://wstein.org/edu/2007/spring/ent/ent-html/node89.html
\Section{Elliptic-curve cryptography \& ECDSA}\label{ecdsa}
As covered in the introduction, Elliptic-curve cryptography (\textbf{ECC}) and ECDS (Elliptic curve digital signature) 
is a fundamental building block of bitcoin. Elliptic curve cryptography relies on 
intractability of calculating the discrete logarithm of a elliptic curve element with 
respect to a publicly known base point.\cite{miller_1986}\cite{nakamoto_bitcoin}\cite{antonopoulos_2017}\cite{ecc_def} Or put another way: It is easy to calculate 
elliptic curve multiplication with multiplicand $n$. But calculating $n$ from the 
resulting point is considered infeasible with sufficiently large curves and multiplicands.\cite{antonopoulos_2017}

%Not sure about the field
An elliptic curve is defined by the equation $Y^2=x^3+ax+b$ and six domain parameters 
$E(p,a,b,G,n,h)$.\cite{Secp256k1_def}\cite{ecc_def} $\textbf{p}$ is the field that the curve is defined over, this 
is usually a very large prime number. The curve being defined over a field simply 
means that the points on the curve fall within $[0, p]$ rather than within the 
real numbers $\mathbb{R}$. In other words the curve is defined over the field 
$\mathbb{F}_{p}$. $\textbf{a}$ and $\textbf{b}$ are whatever number you put into 
the equation. $\textbf{G}$ is the generator point, that is the point on the curve 
that will be used in point multiplication later. $\textbf{n}$ is the order of G. What 
that means is that $n$ is the largest number that $G$ can be multiplied by before 
a point at infinity is produced. $n$ pretty much tells you the limit on how many points 
on the curve that can be generated from $G$. $\textbf{h}$ is the co-factor of the 
curve. It can be calculated as follows: $h=\frac{1}{n}|(E(\mathbb{F}_{p})|$, where 
$|(E(\mathbb{F}_{p})|$ is the order/cardinality of the group of points possible on 
the curve over field $\mathbb{F}_{p}$. $n$ is derived from $G$, $G$ and $p$ should 
be chosen in such a way that $h \leq 4$, preferably $h=1$.

These domain parameters can be chosen manually or you can use predefined parameters. 
Elliptic curves that used predefined domain parameters are called named-curves. 
The named curve used by Bitcoin is called \texttt{Secp256k1}.\cite{Secp256k1_def}\cite{antonopoulos_2017}\cite{nakamoto_bitcoin}

\Subsection{Secp256k1}
\texttt{Secp256k1} is defined with the following domain parameters (hexadecimal):\\\\
$p=\texttt{FFFFFFFF FFFFFFFF FFFFFFFF FFFFFFFF FFFFFFFF FFFFFFFF FFFFFFFE FFFFFC2F}$\\
or alternatively:\\
$p=2^{256}-2^{32}-2^{9}-2^{8}-2^{7}-2^{6}-2^{4}-1$

$a=0$\\
$b=7$

$G=(\texttt{79BE667E F9DCBBAC 55A06295 CE870B07 029BFCDB 2DCE28D9 59F2815B 16F81798},\\ \null\qquad\:\:\: 
\texttt{483ADA77 26A3C465 5DA4FBFC 0E1108A8 FD17B448 A6855419 9C47D08F FB10D4B8})$


$n=\texttt{FFFFFFFF FFFFFFFF FFFFFFFF FFFFFFFE BAAEDCE6 AF48A03B BFD25E8C D0364141}$
$h=\texttt{1}$

\Subsection{Math on the elliptic curve}
Two mathematical operations needs to be defined to operate on the elliptic curve: 
addition and multiplication, as they are defined by \textbf{SECG} (Standards For Efficient CryptoGraphy). In their paper: SEC 1: Elliptic curve cryptography.\cite{ecc_def}

\Subsubsection{Point addition}
Let's say you have to distinct points P and Q that both fall on curve $E(p,a,b,G,n,h)$ 
($Y^2=x^3+ax+b$). 

$$P+Q=R \Rightarrow (X_P, Y_P) + (X_Q, Y_Q) = (X_R, Y_R)$$

$$X_R = \lambda^2-X_P-X_Q$$
$$Y_R = \lambda(x_P-X_R) -Y_P$$

where $\lambda$:

$$\lambda = \frac{Y_Q-Y_P}{X_Q - X_P} \mod p$$

\Subsubsection{Point multiplication}
If P and Q are coincident, meaning that they have the same coordinates the equation 
is slightly different. 

$$P+Q=R \Rightarrow P+P=R \Rightarrow 2P=R$$ 

This could be seen as P being multiplied with scalar 2. Most of the equation is the same 
as with addition, the difference is that:\\
$$\lambda = \frac{(3X^2_P + a)}{(2Y_P)} \mod p$$

\Subsubsection{Faster multiplication with large scalars}
Take $xP=R$ that could be calculated by summing P x times:
$$\sum_{n=1}^{x} P = R$$
This might work fine for smaller numbers but for a very large number, like $x=2^{100}$, it will 
take infeasible amount of time to calculate. Luckily there is a convenient short cut that you 
can take called double and add. 

First remember that: $P+P = 2P \Rightarrow 2P + P = 3P \Rightarrow 4P = 2(2P) \Rightarrow 8P = 2(2(2P))$

Lets say $x=200$ in binary terms this could be written as $x=128+64+8$ or $x=2^7+2^6+2^3$ 
thus $200P=R$ could be written as 
$$2^7P+2^6P+2^3P=R$$ 
which can be expanded to: 
$$2(2(2(2(2(2(2P)))))) + 2(2(2(2(2(2P))))) + 2(2(2P))$$ 
This looks more cumbersome but now instead of 200 calculations you only have to do 19. Or more exactly instead of 200 elliptic-curve additions you have to do 3 elliptic-curve additions and 16 elliptic-curve multiplications


\Subsection{Private and public key}
Just as \texttt{RSA} cryptography, ECC relies on public-private key encryption and signatures. 
The public key can be shared freely to everyone, while the private key should, as the name implies, 
be kept private. Each unique private key has a corresponding public key, through mathematics it 
can be proven that someone holds the private key paired with a certain public key, without actually 
revealing the private key.\cite{antonopoulos_2017}\cite{Secp256k1_def}

\begin{wrapfigure}{r}{0.3\textwidth}
	\begin{center}
		\includegraphics[width=0.4\textwidth]{background/images/key_compression.png}
	\end{center}
	\vspace{-8mm}
	\caption{How to compress the public key in ecc}
\end{wrapfigure}

In ECC \textbf{a private key is a really large number}. Imagine you have curve $E(p,a,b,G,n,h)$ 
and you want to generate a brand new private key k. k could be any number between 0 and $n$. Any 
$k > n$ will produce the exact same public key so that will not work. \textbf{A public key in ECC is 
represented by a point in 2D space}, more specifically a point that falls on the curve.\cite{antonopoulos_2017} To generate a 
public key P from a private key k you perform $kG = P$ as described in the section above.\cite{ecc_def}\cite{Secp256k1_def}

\Subsubsection{Compressed key}
The public key is quite large, with two 256-bit numbers representing coordinates. But there is a 
clever trick we can use to compress the size of the key. Take the \texttt{Secp256k1} curve for 
example ($Y^2=x^3+7$). It is mirrored around the x-axis, meaning that for each x value there 
are two possible y values. Thus a public key can be represented by only it's x value plus a 
prefix telling you which resulting y-value to choose.\cite{antonopoulos_2017}

Note that because y and x is over $\mathbb{F}_{p}$ there is no negative value, instead the y 
value is referred to as even or odd. 




\Subsection{ECDSA}
The main usage of ECC in cryptocurrency is for proving ownership of coins.\cite{antonopoulos_2017}\cite{nakamoto_bitcoin} The proof relies 
on elliptic curve mathematics like before. Lets say \textbf{Alice} has a message $m$ and want 
to send it to \textbf{Bob} and also prove that the message came from her. First of let's 
establish some variables: $k_A$ is the private key belonging to Alice, and from it the public key $P_A$ was generated. Bob knows $P_A$ but not $k_A$ 

The signature generation and validation are as described by \textbf{SECG} (Standards For Efficient CryptoGraphy). In their paper: SEC 1: Elliptic curve cryptography.\cite{ecc_def}

\Subsubsection{Signing}
First calculate the hash of the message:
$$e=HASH(m)$$
If $e$ has a bit-length (numbers in binary representation) that is longer than the bit-length 
of order $n$ of the curve used. $e$ has to be trimmed down so that the bit-lenghts match

Select a cryptographically-secure random number $z$ that falls in the range $[1, n-1]$ and 
calculate a new curve point: $(x_1, y_1) = z \times G$

Calculate $r$ and $s$ such that: $r = (x_1 \mod n)$ and $s = (z^{-1} (e + r \times k_A) \mod n)$. 
if either $r$ or $s$ ends up being 0, generate a new $z$ and try again.

The signature will be the point $(r, s) = S_A$.

\Subsubsection{Signature validation}\label{signature_validation}
If \textbf{Bob} wants to verify that it was actually \textbf{Alice} that signed the message $m$. 
He first has to do a sanity check on the signature $S_A$ to make sure that it is a valid point on 
the curve and that $s$ and $r$ is within the range $[1, n-1]$ etc... 

Calculate the hash $e$ of $m$ the same way as it was done during the signing process. Calculate 
$$w=(s^{-1} \mod n)$$ and $$u_1 = (ew \mod n)$$ $$u_2 = (rw \mod n)$$

From $u_1$ and $u_2$ calculate the point $(x_1, y_1) = u_1 \times G + u_2 \times S_A$ 

The signature is valid if and only if $r = x_1 \mod n$

\Subsection{Addresses}
While using Bitcoin normally you rarely interact with the public keys directly, instead you mostly see  and use something that is called addresses.\cite{antonopoulos_2017} For example if you want to send money to someone you use their bitcoin-address (see section \ref{p2pkh}). What the address really is, is just a obfuscated representation of a public key.  

To transform a public key to an address you first hash it using \texttt{SHA256} then hash the result of that with \texttt{RIPEMD160}, this entire process is called HASH160. After doing HASH160 you encode the resulting bytes in Base58, this is your address.\cite{antonopoulos_2017}

HASH160($P_A$) = RIPEMD160(SHA256($P_A$))\\
$A_A$ = BASE58(HASH160($P_A$))

For example:

Compressed public key:\\
\texttt{03b2319bf63ca8959794f79056283150f1b49d577b2c48bd9e4a18d616e6f99bb4}

Hash160 of public key:\\
\texttt{76a9147ecfe7db089f521c3f49de0ee866f8e0fee97d1788ac}

Base58 of public key hash (Address):\\
\texttt{ms5UVmKaaz8kiL7DbMA1NG3t6T4C4GmGzG}

\Subsection{Homomorphism}\label{homomorphism}
Elliptic curves has a attribute called homomorphism. It will not be covered in depth, but what it can be used for is that two public keys ($P_1$, $P_2$) can be combined to form a third public key ($P_R$). If $P_1$ and $P_2$ was generated with corresponding private keys $k_1$ and $k_2$ then they can be combined to form the private key that generates $P_R$ ($k_R$).\cite{miller_1986}\cite{bolt}

There are several ways of calculating a new homomorphic key, this is the one used in later implementations:

Let's say you have private key k with corresponding public key P. And another temporary private key $k_c$ with corresponding public key $P_c$. To generate the homomorphic public key ($P_h$) you can use the following equation (All the operations in equation \ref{generate} below are in elliptic-curve addition and multiplication, The second (\ref{resolve}) uses regular operators):

\begin{equation}\label{generate}
Generate(P, P_c) = P * SHA_{256}(P || P_c) + P_c * SHA_{256}(P_c || P) = P_h
\end{equation}

You now have Public key $P_h$. To figure out the Private key correlated with this public key you do the following equation:

\begin{equation}\label{resolve}
Resolve(k, k_c) = k * SHA_{256}(P || P_c) + k_c * SHA_{256}(P_c || P) = k_h
\end{equation}

\Subsubsection{Revocation keys}
Imagine Alice and Bob wanting a public key that neither hold the private key to, that could be revealed at a later date. Alice holds the key set ($k_A$, $P_A$), and Bob has the set ($k_B$, $P_B$). Bob creates another key set ($k_T$, $P_T$) and sends $P_T$ to Alice.

Alice generates a a new public key ($P_R$) from $P_A$ and $P_T$ with equation \ref{generate}. Now the public key $P_R$ could be used for whatever purpose needed, even though neither party knows the private key $k_c$ at this point as calculating it requires both $k_A$ and $k_T$ and neither party holds both. At a later date Bob could reveal $k_T$ to Alice, she could then use equation \ref{resolve} to determine $k_R$.\cite{bolt}

This method is used later in a concept called revocable deliveries, see section \ref{breach_remedy}. In that case ($k_T$, $P_T$) key set would be called commitment point keys, and ($k_R$, $P_R$) would be called revocation keys.\cite{bolt}

\Section{Script}

Script is the name of the programming language used in Bitcoin and its derivatives. It was not used to write the Bitcoin implementation but rather it is what makes transactions in Bitcoin so versatile. This section however will not focus on where or how Bitcoin uses Script, but rather on how Script itself functions.

\begin{wrapfigure}{l}{0.4\textwidth}
	\begin{center}
		\includegraphics[width=0.4\textwidth]{background/images/script.png}
	\end{center}
	\vspace{-8mm}
	\caption{Execution of simple program, to the left is the stack, to the right is the program with execution pointer}
	\label{fig:script}
\end{wrapfigure}

Script is a forth-like stack based language that is not turing-complete. Turing-completeness means that a language can do anything that the imaginary turing-machine could do, in other words it basically means is that the language can do any mathematically sound operation. Script is \textbf{NOT} turing-complete on purpose, more on why can be found in the section about transactions (\ref{transactions}). A good example is loops, in most languages there is some sort of structure that allows for a piece of code being executed repeatedly. In Scipt this is strictly disallowed, as it has neither for-loops or while-loops. 

As mentioned earlier. Script is a stack-based language. That means that as the language executes it uses a stack to store data and variables. Do not confuse the term with the heap and stack from regular programming language discourse. In script there is no heap, instead the stack is the only form of memory, and it acts just as you would expect from a stack, to add a value you have to push it to the stack and to read a value you have to pop it from the stack.

The language it self is quite basic. It relies on operation codes (op codes) the size of a single byte. Most operations pop values from the stack, does something with the values, then pushes the result back on the stack. When Script is written out on paper many op codes and values are excluded, because they are implicit. For example: 

\texttt{OP\_PUSHDATA1 4 FFFFFFFF} 

This operation pushes 4 bytes to the stack, the bytes all have the hexadecimal value \texttt{FFFFFFFF}. But usually when this operation is written out it is shorten to just:

\texttt{FFFFFFFF} 

Whenever a hexadecimal value appears in the code it is implicit that that value is pushed to the stack in the form of bytes. Here is annother program:

\texttt{4 5 OP\_ADD 9 OP\_EQUAL}

This simple program will push 4 and 5 to the stack, \texttt{OP\_ADD} pops two values from the stack (4 and 5) adds them together and pushes the result back on the stack. Then a 9 is pushed to the stack. \texttt{OP\_EQUAL} pops two values form the stack and pushes a 1 if they were equal, 0 otherwise. In this case a 1 will be left on the stack as $4 + 5 = 9$. Figure \ref{fig:script} visualizes what happens at each step of execution.

\Subsection{Complex operations}
There are some operations in Script with slightly higher complexity, which do not act like the others. One of them is \texttt{OP\_VERIFY}, this will perform a verify check on the script. It will pop one value from the stack, and if that value equates to false execution will end immidietly and the entire script will be marked as invalid (see section \ref{script_valid}). If the value equals true execution will continue where it left off. Some operations are combinations with the verify check, like for example \texttt{OP\_EQUALVERIFY}. This is equal to writing \texttt{OP\_EQUAL OP\_VERIFY}, meaning that it first does an equal check then verifys the result.

There are several operations for checking signatures. These are not so complex in terms of what they do in the script. Their implementation is quite complex however. They reliy on outside information that is not present in the script to check the validity of a signature. The two most used are \texttt{OP\_CHECKSIG} and \texttt{OP\_CHECKMULTISIG}. These will not be covered in full in this section as they are more related to transactions so see section \ref{transactions} for a full explanation. 

\Subsection{Valid and invalid scripts}\label{script_valid}
At the end of execution a script is marked as either valid or invalid. A script is invalid for the following reasons:

\begin{itemize}
	\item The stack is empty
	\item There are more than one values on the stack
	\item The only value on the stack equates to false
	\item A VERIFY check fails sometime during execution.
\end{itemize}

A script is valid if and only if there is one value on the stack, and it is not equal to false.

\Subsection{Bug in Script}
An interesting bit of trivia is that there is a bug in the language implementation. More specifically with the operation \texttt{OP\_CHECKMULTISIG}. The bug makes it so this operation pops one more value from the stack than it is supposed to. This was not discovered until the network had been running for a while. It can't be easily fixed as it is now a part of the consensus rules. All implementations of Bitcoin node has to implement the bugged version of this op code otherwise consensus on the validity of transactions will break.

A fix to this bug would require a hardfok, see section \ref{soft_hard_fork}, and has so far been considered not worth it.
%https://en.bitcoin.it/wiki/Transaction
%mastering bitcoin
\Section{Transactions}\label{transactions}
Transactions in Bitcoin are not as straight forward as you might expect a transaction to be. A transaction contains a list of inputs and a list of outputs as well as some metadata like version number and lock-time. 

\begin{figure}[H]
	\centering
	\includegraphics[width=1.0\textwidth]{background/images/transaction_basic.png}
	\caption{4 example transactions and how inputs are connected to outputs}
	\label{fig:transaction_input_output}
\end{figure}

In simplified terms an output could be seen as the destination of a transaction, in other words it says how much and to whom the transaction is sent to. An input is a reference to a previous output. The inputs take the money from the outputs they reference and that money is used to fun the new outputs. 

The inputs and outputs is where Script comes into the picture. Both outputs and inputs contains an incomplete script, together however they complete the script. The script in an output could be seen as a challenge, and the script in the input is the response. When a transaction is tested for validity the input script is appended to the script in the output and is executed. If the script comes out as valid the transaction is also valid. Here is a basic example: Let's say Alice wants to send a transaction to whoever can answer the very complex equation $4+3$. Her transaction output would contain the script:

\texttt{4 3 OP\_ADD OP\_EQUALS}

If this is executed as is it is invalid. But let's say Bob knows the answer to the equation he can then create a new transaction where the input contains the script: 

\texttt{7} 

Just as before this script is not valid by itself. But then the transaction is checked for validity the input will be appended to the start of the output script forming the following: 

\texttt{7 4 3 OP\_ADD OP\_EQUALS}

Which is a valid script, thus bobs new transaction is also valid and he may spend the money as he see fits. 

Obviously most transactions on the Bitcoin blockchain are not this simple. The most common form of transaction contians a script called P2PKH which stands for Pay to public key hash. Before we can go into details on this one however we first need to know about how signatures and sighash work in script and transactions.

\Subsection{Signatures and sighash}
Section \ref{ecdsa} covers public keys and signatures in depth.

Perhaps the most important operation in script is the \texttt{OP\_CHECKSIG} operation and its cousins. \texttt{OP\_CHECKSIG} pops two values from the stack, if the script is correctly implemented these two values should be the public key and a signature that was signed with the private key that was used to create the public key. 

The question is: what is signed when the signature is created? Broadly speaking it is the hash of the transaction that is trying to spend the output, this is not entirely correct however. Appended to the signature that is a flag called \textbf{sighash}. The value of sighash tells the script interpreter what hash was signed during the creation of the signature. There are 4 types of sighash implemented:

\Subsubsection{SIGHASH\_ALL}
This can be considered the default sighash, if it is not specified it can safely be assumed that this type was used. This simply means that the entire transaction is signed with all outputs and all other inputs.


\Subsubsection{SIGHASH\_NONE}
This one signs the transaction butwithout the outputs, it could be thought of as ''I don't care where the money goes''

\Subsubsection{SIGHASH\_SINGLE}
All outputs are removed except the output with the same index as the input that is being signed, then that transaction is signed.

\Subsubsection{SIGHASH\_ANYONECANPAY}
Signs the transaction with all the outputs but none of the other inputs. This basically means ''The money has to go here, but I don't care if someone else want to fund this transaction also''

\Subsubsection{More detailed process}
On the next page the entire signing process for SIGHASH\_ALL is detailed. This is how it is performed in the actual implementation:
\newpage
\centerline{\includegraphics[width=1.35\textwidth]{background/images/checksig_in_detail.png}}
\newpage



\Subsection{Pay to public key hash (P2PKH)}
\Subsection{Pay to script hash (P2SH)}
\Subsection{Timelock and sequence}

\Section{Segregated Witness}
\Section{Lightning network}

\Section{On-chain Atomic swaps}
On-chain atomic swaps is a method where two parties can exchange cryptocurrencies between different chains in an atomic way, in this case it means that none of the parties can cheat the other and no matter what state in the exchange-process the swap is either completed or reversed entirely to the pre-exchange state. The on-chain distinction comes from the fact that this type of atomic swap occurs entirely on the chains (There is no off-chain transaction in the process).

There are several ways atomic swaps can be performed, scripts, signatures and time locks can take several different combinations in transactions and still retain atomicity. The method that will be described in this chapter will be a P2SH, pay to contract type. This swap is done with the help of time locked contracts constructed with scripts. 

\Subsection{The contracts}
A swap contract contains certain components. The initial component is the secret $x$ that is generated by the initial contract creator. The secret is not included in the initial contract however, this is what is included:
\begin{itemize}
	\item Byte string $h$ \quad - \quad Hash of secret $x$
	\item Integer $T$ \quad\quad - \quad Contract expiration time as timestamp
	\item Address $B$ \quad - \quad Redeemer address
	\item Address $A$ \quad - \quad Refund address
\end{itemize}

The contract uses branching provided by the operations \texttt{OP\_IF}, \texttt{OP\_ELSE} and \texttt{OP\_ENDIF}. The branch that is taken during execution is decided by whoever is trying to spend the output. One branch pays to the redeemer address, for it to be valid the secret has to be revealed. The other branch pays to the refund address, but it can only be valid if the current time is equal to or greater than the timestamp T.

\Subsubsection{Counter party contract}
The contract described above is the one constructed by and broadcast by whoever wants to initialize the exchange. A very similar contract has to be constructed and broadcast by the counter party, this contract however contains a few changes. The only variable that remains unchanged is $h$. $T$ has to be be a value between the current time and $T$ from the old contract, denoted $T/2$. Redeemer and refund addresses has to be changed so that the redeemer is whoever constructed the initial contract and the refund address is yourself.

\Subsection{The process}
Imagine Alice and Bob wants to swap cryptocurrencies. Alice has Bitcoin and wants Litecoin, Bob vice versa. Alice is the initiator in this case. The process would be the following:
\begin{enumerate}
	\item Alice generates a new secret and then constructs a contract from the variables needed as stated above.
	\item She broadcasts a p2sh transaction (to the Bitcoin blockchain) that pays to the constructed contract, called the contract transaction. Alice then sends the contract and the contract transaction txid to Bob.
	\item Bob fetches the transaction from the Bitcoin blockchain and validates all the variables, as well as the contract to contract hash etc\dots
	\item If all seems to be in order, Bob can construct his own contract using the same secret hash h as Alice did in her contract. This time however the timelock is set to $T/2$. He then broadcasts a transaction paying to the contract on the Litecoin network. Just as Alice did, Bob sends the contract and txid to Alice.
	\item Alice validates all the data the same way that Bob did in step 3. 
	\item If all seems to be in order. Alice could claim the Litecoins from Bobs contract transaction by making a transaction spending the output. The input that validates the output script must contain the secret x to be valid.
	\item Meanwhile Bob monitors the Litecoin chain for someone spending his contract transaction. If somebody does it means that they know the secret $x$. Any information in the blockchain can be extracted, Bob needs to know $x$ to spend Alice's contract transaction. If somebodies spends his contract transaction he can extract $x$ from that and spend the Bitcoin side transaction.
\end{enumerate}

\Subsubsection{Visualizer}
The next page contains a diagram visualizing the process described above. T is set to 48 hour after initialization in the example, thus T/2 would be 24 hours.

\Subsection{Atomicity}
Easiest way to show the atomicity of the swap is to show what happens if one party stops cooperating during any step in the process. The two trivial cases are before any step in the process, then no transaction has been made and the state did not change at all. If one becomes uncooperative at the end of the process the state change has already been completed and it does not matter what anyone does. The non-trivial cases is when someone becomes uncooperative mid process.\\

The first case we will look at is if Bob stops responding after step 2. Meaning he never broadcasts his side of the contract. Then Alice simply never reveals $x$, Bob needs $x$ to claim the Bitcoins. Alice then just waits until timelock $T$ has passed and refunds the transaction. After the refund the state has been reset to the initial state.

The second case is if Alice becomes unresponsive after Bob has broadcast his side of the contract. Bob can not do anything until he knows $x$ or the timelock ($T/2$) of his transaction expires. If Alice does nothing Bob will never know $x$ and thus has to wait to refund his transaction. Presumably Alice does the same after timelock $T$ expires. The state has been reset to the initial one.

\newpage
\centerline{\includegraphics[width=1.35\textwidth]{background/images/atomic_swap_flow_large.png}}
\newpage

\Subsection{How the swap could fail}
The atomicity of the swap depends on each party acting rationally and not doing any mistakes that compromises the process. For example if Alice broadcasts her side of the contract and then reveals $x$ to Bob thorough some alternative communication channel then Bob could claim the Bitcoins without doing his part of the deal. 

Malformed contracts could also lead to money being stolen. This is why the validation steps are very important. For example Alice could maliciously set her contract to expire earlier than Bobs if Bob is not careful.

Actors in the scenario doing nothing also compromises the swap. If Alice waits until after $T/2$ expires before attempting to claim the Litecoins she is risking out on losing both the currencies. Bob could monitor the transaction pool, seeing what $x$ was, then sneak in his own refund transaction reclaiming his Litecoins (This is possible as the refund branch of the contract now is unlocked). Then also claiming the Bitcoins before $T$ expires.

\Section{Off-chain atomic swaps}

\chapter{Atomic swaps}

\Section{On-chain Atomic swaps}
On-chain atomic swaps is a method where two parties can exchange cryptocurrencies between different chains in an atomic way, in this case it means that none of the parties can cheat the other and no matter what state in the exchange-process the swap is either completed or reversed entirely to the pre-exchange state. The on-chain distinction comes from the fact that this type of atomic swap occurs entirely on the chains (There is no off-chain transaction in the process).

There are several ways atomic swaps can be performed, scripts, signatures and time locks can take several different combinations in transactions and still retain atomicity. The method that will be described in this chapter will be a P2SH, pay to contract type. This swap is done with the help of time locked contracts constructed with scripts. 

\Subsection{The contracts}
A swap contract contains certain components. The initial component is the secret $x$ that is generated by the initial contract creator. The secret is not included in the initial contract however, this is what is included:
\begin{itemize}
	\item Byte string $h$ \quad - \quad Hash of secret $x$
	\item Integer $T$ \quad\quad - \quad Contract expiration time as timestamp
	\item Address $B$ \quad - \quad Redeemer address
	\item Address $A$ \quad - \quad Refund address
\end{itemize}

The contract uses branching provided by the operations \texttt{OP\_IF}, \texttt{OP\_ELSE} and \texttt{OP\_ENDIF}. The branch that is taken during execution is decided by whoever is trying to spend the output. One branch pays to the redeemer address, for it to be valid the secret has to be revealed. The other branch pays to the refund address, but it can only be valid if the current time is equal to or greater than the timestamp T.

\Subsubsection{Counter party contract}
The contract described above is the one constructed by and broadcast by whoever wants to initialize the exchange. A very similar contract has to be constructed and broadcast by the counter party, this contract however contains a few changes. The only variable that remains unchanged is $h$. $T$ has to be be a value between the current time and $T$ from the old contract, denoted $T/2$. Redeemer and refund addresses has to be changed so that the redeemer is whoever constructed the initial contract and the refund address is yourself.

\Subsection{The process}
Imagine Alice and Bob wants to swap cryptocurrencies. Alice has Bitcoin and wants Litecoin, Bob vice versa. Alice is the initiator in this case. The process would be the following:
\begin{enumerate}
	\item Alice generates a new secret and then constructs a contract from the variables needed as stated above.
	\item She broadcasts a p2sh transaction (to the Bitcoin blockchain) that pays to the constructed contract, called the contract transaction. Alice then sends the contract and the contract transaction txid to Bob.
	\item Bob fetches the transaction from the Bitcoin blockchain and validates all the variables, as well as the contract to contract hash etc\dots
	\item If all seems to be in order, Bob can construct his own contract using the same secret hash h as Alice did in her contract. This time however the timelock is set to $T/2$. He then broadcasts a transaction paying to the contract on the Litecoin network. Just as Alice did, Bob sends the contract and txid to Alice.
	\item Alice validates all the data the same way that Bob did in step 3. 
	\item If all seems to be in order. Alice could claim the Litecoins from Bobs contract transaction by making a transaction spending the output. The input that validates the output script must contain the secret x to be valid.
	\item Meanwhile Bob monitors the Litecoin chain for someone spending his contract transaction. If somebody does it means that they know the secret $x$. Any information in the blockchain can be extracted, Bob needs to know $x$ to spend Alice's contract transaction. If somebodies spends his contract transaction he can extract $x$ from that and spend the Bitcoin side transaction.
\end{enumerate}

\Subsubsection{Visualizer}
The next page contains a diagram visualizing the process described above. T is set to 48 hour after initialization in the example, thus T/2 would be 24 hours.

\Subsection{Atomicity}
Easiest way to show the atomicity of the swap is to show what happens if one party stops cooperating during any step in the process. The two trivial cases are before any step in the process, then no transaction has been made and the state did not change at all. If one becomes uncooperative at the end of the process the state change has already been completed and it does not matter what anyone does. The non-trivial cases is when someone becomes uncooperative mid process.\\

The first case we will look at is if Bob stops responding after step 2. Meaning he never broadcasts his side of the contract. Then Alice simply never reveals $x$, Bob needs $x$ to claim the Bitcoins. Alice then just waits until timelock $T$ has passed and refunds the transaction. After the refund the state has been reset to the initial state.

The second case is if Alice becomes unresponsive after Bob has broadcast his side of the contract. Bob can not do anything until he knows $x$ or the timelock ($T/2$) of his transaction expires. If Alice does nothing Bob will never know $x$ and thus has to wait to refund his transaction. Presumably Alice does the same after timelock $T$ expires. The state has been reset to the initial one.

\newpage
\centerline{\includegraphics[width=1.35\textwidth]{background/images/atomic_swap_flow_large.png}}
\newpage

\Subsection{How the swap could fail}
The atomicity of the swap depends on each party acting rationally and not doing any mistakes that compromises the process. For example if Alice broadcasts her side of the contract and then reveals $x$ to Bob thorough some alternative communication channel then Bob could claim the Bitcoins without doing his part of the deal. 

Malformed contracts could also lead to money being stolen. This is why the validation steps are very important. For example Alice could maliciously set her contract to expire earlier than Bobs if Bob is not careful.

Actors in the scenario doing nothing also compromises the swap. If Alice waits until after $T/2$ expires before attempting to claim the Litecoins she is risking out on losing both the currencies. Bob could monitor the transaction pool, seeing what $x$ was, then sneak in his own refund transaction reclaiming his Litecoins (This is possible as the refund branch of the contract now is unlocked). Then also claiming the Bitcoins before $T$ expires.
// sectuon goes here

atomic swaps over a payment channel is possible, often refered to as offchain atomic swap. the mechanisms that makes it work may not be as clear as one that takes place entirely onchain. 
to clear somethings up before moving on: both parties that are performing the swap needs to have a functioning node on the respective network 


\chapter{Implementation of atomic swaps}
So finally we get to the implementation part of the presentation. Becauce I did
two different types of atomic swaps I have decided to split this part up into two parts.
The first parts cover what is known as on-chain atomic swaps, and the second part
will be the bit more complex off-chain atomic swaps.

\Section{on-chain atomic swap}
THere are several ways to implement a on-chain atomic swap. The one I choose 
is probably the easiest to understand. Before I go into implementation specifics 
it would be good to know how this process actually works. 

Imagine two people, Alice who has Bitcoins and wants litecoins, and Bob who
has litecoins and wants bitcoins. Here a swap is possible. THe methods people
start thinking about right away could be for example some sort of third-party
that ahndles the swap. Or maybe Alice just sends the Bitcin to Bob and hopes
that he doesnt take the money and run. Overall the problem he is trust. 
Even with a third-party there is a chance that Bob and the third-party
are cooperating to steal from Alice. With the help of programmable contracts 
however, we now have a method of swapping where the only thing you have to 
trust is numbers.

The process of an on-chain atomic swap is as follows:
Alice and Bob agrees to do a swap, 1 bitcoin for 10 litecoins. They also 
decide that Alice should be the one to initiate the exchange. 
To start off, Alice generates a random bytestring that will act as a 
pre-image, let's call this R. She hashes this pre-image and produces H_R.  
With te help of H_R she constructs a new swap contract with the following clauses. Pay 1
bitcoin to Bob if he can provide the pre-image of H_R (R). If Bob does not claim 
this output wihin 48 hours, refund the full amount to Alice. 

Alice broadcasts this contract transaction to the bitcoin blockchain and
notifies Bob of doing so, she also sends the unhahsed contract to Bob. 
Bob can then fetch the transaction from the blockchain, he then makes
sure that the contract hash matches the one on the chain (P2SH) and
he validates all the details of the contract. 


\chapter{Results \& Proposal}
In this section I will talk about the results I got when performing the types of atomic swaps detailed in previous chapters, and from these and my experience I will make a few proposals on how atomic swaps could  be standardized into clearly defined protocols and procedures. 

\Section{On-chain atomic swaps}
In my implementation I created a scenario where two pararties swapped bitcoins for litecoins, on their respective testnets. 
The swap was tested for all possible outcomes and possible actions by the actors. And in all cases the swap acted in an atomic way, meaning that either the exchange tock place fully or the state was eventually reset to the original. 
As expected an onchain atomic swap takes some time to perform. With the values for the timelocks used in my experiment the swap could take up to 48 hours to complete in a worst case scenario. The timelock values could be decreased at the cost of increased risk of failure. 

Under the assumption that the contract transaction is included in the next block the



\chapter{Discussion \& Future research}
Overall I am happy with the work. The project started out with goals that were different than what it ended up being. What I am most unhappy about is that I realized so late into the project a much better goal, leading to a lot of time being wasted on experiments that were not particularly useful. 

If I could do the project over again my focus would have been more on actually developing the swapping plugin described in chapter 5. I feel personally that this would have been much more fulfilling, and it would have been a better demonstration of my actual proposal. As it is not the experiments and the proposal feels a bit disconnected, what I mean is that I could have reached the conclusion without performing most of the experiments. Even though my implementation was not entirely necessary I am still happy I did them, as there is very little information about atomic swaps especially the off-chain form. It was nice to get first hand experience with actual development using bitcoin tools and libraries, and experience with programmed contracts. 

I plan to at least give it a try to implement the plugin, but it will be during the coming summer and outside the scope of this thesis. 

\Section{Future research}
There is plenty of 

\twocolumn


\onecolumn
% Bibliography
\bibliographystyle{plain}
\bibliography{reference}


\end{document}
